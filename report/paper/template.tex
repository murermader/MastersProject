%%%%%%%%%%%%%%%%%%%%%%%%%%%%%%%%%%%%%%%%%%%%%%
%%%%%%%%%%%%%%%%%%%%%%%%%%%%%%%%%%%%%%%%%%%%%%
%%% Master Thesis Template by Fabian Schär %%%
%%%%%%%%%%%%%%%%%%%%%%%%%%%%%%%%%%%%%%%%%%%%%%
%%%%%%%%%%%%%%%%%%%%%%%%%%%%%%%%%%%%%%%%%%%%%%

%%%%%%%%%%%%%%%%%%%%%%%%%%%%%%%%%%%%%%
%%% Packages and Document Settings %%%
%%%%%%%%%%%%%%%%%%%%%%%%%%%%%%%%%%%%%%

\documentclass[12pt,a4paper,titlepage,oneside,english]{article}

%%% Main Packages %%%
\usepackage[english]{babel}
%\usepackage[ngerman]{babel} % Use this option for German settings.
\usepackage[T1]{fontenc}
\usepackage[utf8]{inputenc}
\usepackage{lmodern}

%%% Additional Packages %%%
\usepackage{cite}
\usepackage{framed}
\usepackage{graphicx}
%\usepackage[german]{fancyref}
\usepackage[german,hidelinks]{hyperref} %hidelinks
\usepackage{multirow}
\usepackage[round]{natbib}
\usepackage{setspace}
\usepackage{geometry}
\usepackage{pst-all} % Not working with Sweave!!!

%%% Math Packages %%%
\usepackage{amsmath}
\usepackage{amstext}
\usepackage{amssymb}
\usepackage{theorem}
\usepackage{epsfig}
\usepackage{longtable}

% I added these
\usepackage{microtype}
\usepackage{pgfgantt}
\usepackage{xcolor}

%%% Layout Specifications %%%
\geometry{a4paper, top=35mm, left=40mm, right=40mm, bottom=45mm,
headsep=10mm, footskip=12mm}

%%% Parskip Settings %%%
\setlength{\parskip}{3mm}
\setlength{\parindent}{0mm}

%%% Document Specifications %%%
\title{Bias and Fairness in Digital Archival System}
\author{Rafael Biehler}


%%%%%%%%%%%%%%%%%%
%%% Title Page %%%
%%%%%%%%%%%%%%%%%%

\begin{document}
%\begin{titlepage}
\begin{center}
\vspace{1em}
%\large{Seminar Paper}\\
%\large{Bachelor Thesis}\\
\large{Master's Project}\\
\huge Bias and Fairness in Digital Archival System \\
\Large \vspace{1em}
Rafael Biehler
\end{center}

\vspace{1em}
\normalsize
\begin{flushleft}
Supervised by:\\ 
Prof. Dr. Heiko Schuldt \\
Fynn Farouz \\ 
\end{flushleft}

\vspace{1em}
\onehalfspacing
\begin{center}
\section*{Abstract}
\end{center}

Todo

\pagenumbering{gobble}

\newpage
\pagenumbering{Roman}
\tableofcontents

\vfill
\begin{center}
\includegraphics[width=4cm]{../assetlib/images/logo_cif.png}
\end{center}
\singlespacing
\vspace{-1.5cm}

\section*{Plagiatserklärung}

Ich bezeuge mit meiner Unterschrift, dass meine Angaben über die bei der Abfassung meiner Arbeit benutzten Hilfsmittel sowie über die mir zuteil gewordene Hilfe in jeder Hinsicht der Wahrheit entsprechen und vollständig sind. Ich habe das Merkblatt zu Plagiat und Betrug vom 22. Februar 2011 gelesen und bin mir der Konsequenzen eines solchen Handelns bewusst.

Rafael Biehler

\newpage
\onehalfspacing
\pagenumbering{arabic}

\section{Time Management}

The following chart depicts my schedule from today (15.12.2023) to the submission date (12.02.2024). Research and Writing are spread around to each three weeks, because in this time I will be studying and working on other lectures as well. Writing and the final review process are relatively short, as I will be able to work full-time. 

\begin{ganttchart}[
    hgrid,
    vgrid,
    x unit=2.25mm,
    bar height=0.7,
    time slot format=little-endian,
    inline
    ]{11.12.2023}{11.02.2024}
    \gantttitlecalendar{year, month, week} \\

    \ganttbar{Research}{11.12.2023}{31.12.2023} \\

    \ganttbar{Experiments}{01.01.2024}{21.01.2024} \\

    \ganttbar{Writing}{22.01.2024}{04.02.2024} \\

    \ganttbar{Review}{05.02.2024}{11.02.2024} \\
\end{ganttchart}

\subsection{Milestones}
\begin{enumerate}
    \item \textbf{Research:} Be familiar with the basics of bias and fairness metrics. Know about the most recent methods of evaluating bias and fairness in similar contexts, like generative AI models. At the end of this block, I should have a clear plan of how I should structure my experiments.
    \item \textbf{Experiments:} Write code to test my plan in action. Calculate numbers which could indiciate performance. Run more tests to challenge my resulsts, like a trial against randomly generated content.
    \item \textbf{Writing:} Write the report for the search.
    \item \textbf{Review:} Review the report together with Fynn to make sure it is acceptable for handing in.
\end{enumerate}

\section{Introduction}

% Introduction Outline for Research Paper

% Part 1: About the Topic + Motivation
% 1. Opening Statement: Start with a broad context, engaging the reader and providing background.
% 2. Narrowing Down: Focus on your specific topic and mention relevant research.
% 3. Research Gap: Identify the gap or problem your research addresses and its significance.
% 4. Motivation: Explain why this topic matters, its impact, and its broader relevance.
% 5. Research Question/Hypothesis: State your research question or hypothesis.
% 6. Purpose of Study: Describe the purpose and how it addresses the identified gap.

In the context of Galleries, Libraries, Archives, and Museums (GLAM), a prevalent challenge is the imbalance between the volume of work and the available staffing. This often leads to a substantial amount of image data being left unlabeled or uncategorized. To mitigate this issue, there is increasing interest in the use of multimedia retrieval systems that enable querying image data by using natural language descriptions. 

% This is not how it works.
%The image retrieval system works by first classifying images into categories, which can then be searched with text. How bias and fairness in classification tasks can be measured is still not entirely clear. How these feature extraction models perform under the 
% Multimedia Retrieval systems are based on closing the gap between different modalities like text and image data by using Feature Extraction Models, which can introduce biases and fairness issues in various different ways. Biases can have significant impact on the results, which are often negative and unwanted. Understanding and measuring these biases, which can have a significant impact on the user, is important to be able to be aware of any issues while also introducing mitigations.

Feature Extraction Models are used to represent image and text data in a common format. This process may introduce biases and fairness issues, which can impact the results in unintended and often undesired ways. It is not yet clear how to best detect unfairness issues for classification tasks.

% In the GLAM context, a unfair image classification could lead to discrimination against a section of the population. To illustrate this problem, we will look into the discrimination of the First Nations people of Canada in the GLAM context, to see if and how much they are affected by any biases.

We will introduce a new method that tries to be able to accurately detect Bias and Fairness issues. We will be working with a GLAM dataset that contains labeled photographs, which we will use to accurately determine potential biases against First Nations people. 

% We will try to introduce a method for accurately detecting bias, which we will also compare to other already well-known techniques. This method will be tested on a GLAM dataset, to accurately determine potential biases against First Nations people.

%We plan to establish a methodology for assessing the fairness and bias using our dataset. This allows us to measure and compare different models and model versions in the future. 

% Part 2: Contents of the Paper
% (To be written later)
% \textcolor{lightgray}{TODO: Write exact content of paper} 


\section{Background}

Todo

\section{Methods}

Todo

\section{Discussion}

Todo

\newpage
\setcounter{page}{1}
\pagenumbering{roman}
\onehalfspacing
\addcontentsline{toc}{section}{References}
\bibliography{mybib}
\bibliographystyle{agsm}

\end{document}
